\documentclass[12pt]{article}

% Packages
\usepackage{hyperref}
\usepackage[utf8]{inputenc}

% Author and project setup
\author{Chelcea Claudiu-Marian}
\title{Dezvoltarea de jocuri video - ce motor grafic s\^{a} folosesc?}
\date{03.18.2021}
\renewcommand{\contentsname}{Cuprins}
\renewcommand{\refname}{Bibliografie}
\setlength{\arrayrulewidth}{1mm}
\setlength{\tabcolsep}{18pt}
\renewcommand{\arraystretch}{1.5}

% Main code
\begin{document}

% Intro page
\maketitle

% Contents page
\newpage
\tableofcontents

% First page
\newpage
\section{De ce un motor grafic?}
\hspace{10pt}
Un motor grafic este un sistem conceput pentru crearea și dezvoltarea de {\it jocuri video}. \par Există mai multe motoare de joc, care sunt proiectate să funcționeze pe console de jocuri video și calculatoare personale. \par Funcționalitatea de bază oferită de obicei de un motor grafic include un motor de randare (engleză renderer) pentru grafică {\bf 2D} sau {\bf 3D}, un motor de fizică sau de detectare a coliziunilor (și răspunsul la coliziune), sunet, scripting, animație, inteligență artificială, în rețea, streaming, memorie de management, suport de localizare, etc. \par Procesul de dezvoltare a jocului este de multe ori economisit, în mare parte, prin reutilizarea / adaptarea unui motor asemănător pentru a crea jocuri diferite.

% Second page
\newpage
\section{Ce motoare grafice sunt populare?}
\hspace{5pt} Lista celor mai populare 10 motoare grafice, conform \href{https://www.gamedesigning.org/career/video-game-engines/}{GAMEDESIGNING}:
\begin{itemize}
  \item Unreal Engine.
  \item Unity
  \item GameMaker
  \item Godot
  \item AppGameKit
  \item CryEngine
  \item Amazon Lumberyard
  \item RPG MAKER
  \item LIBGDX
  \item URHO3D
\end{itemize}

O analiză de la Gamasutra a constatat că Unreal și Unity sunt printre cele mai populare motoare de joc. Această analiză s-a bazat pe jocuri lansate pe Steam și Itch.io.

% Third page
\newpage
\section{Diferen\c{t}e dintre Unity si Unreal Engine}
\subsection{Unity}

Unreal Engine este frecvent utilizat deoarece este:

O platformă de creare 3D deschisă și avansată în timp real.
Folosit pentru a produce jocuri în mai multe genuri.
Integrat cu instrumente cheie de dezvoltare a jocurilor, inclusiv IDE-uri, instrumente grafice și controlul versiunilor.
\subsection{Unreal Engine}
Unity este frecvent utilizată deoarece este:

Un motor de joc multiplataforma, cu suport pentru peste 25 de platforme.
O platformă de dezvoltare 3D în timp real.
Integrat cu instrumente cheie de dezvoltare a jocurilor, inclusiv IDE-uri, instrumente grafice și controlul versiunilor.

% Fourth page
\newpage
\section{B\u{a}t\u{a}lia final\u{a}}
\begin{tabular}{ |p{3cm}|p{3cm}|p{3cm}|  }
\hline
\multicolumn{3}{|c|}{Unreal Engine versus Unity} \\
\hline
& Unity & Unreal Engine \\
\hline
%Pret & \text{Gratuit | 125$ pe luna | 5\% din profit} & \text{Gratuit | 5\% din profit} \\
Usurinta de utilizare & Interfata simpla, mai multe surse de invatare   & Interfata grea, resurse de invatare putine \\
Utilizare & BluePrint ( clase ) - necesita programare & VisualScripting sau Coding - nu necesita programare \\
Limbaj de programare  & C-sharp &  Cplusplus \\
Grafica & Grafica este buna spre foarte buna & Cele mai bune grafici oferite de un motor grafic \\
\hline
\end{tabular}

% Bibliography
\newpage
\begin{thebibliography}{}
\bibitem{text1} 
\href{https://ro.wikipedia.org/wiki/Motor_grafic}{Wikipedia: motor grafic link}
\bibitem{text2} 
\href{https://www.gamedesigning.org/career/video-game-engines/}{Gamedesigning: website link}
\bibitem{text2} 
\href{https://www.gamasutra.com/}{Gamasutra: website link}
\end{thebibliography}

\end{document}
